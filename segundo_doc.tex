% Clase del documento
\documentclass[a5paper, 12pt]{article}

% Paquetes
\usepackage[utf8]{inputenc}
\usepackage[spanish]{babel}
\usepackage[top=2cm, left=2cm]{geometry}
\usepackage{amsmath, amssymb, amsfonts, latexsym}
\usepackage{graphicx}
\usepackage{color}
\usepackage{multicol}
\usepackage{nicefrac}

% Comandos
\parindent = 0mm

\author{Milton Torres \and Andrés Mt}

\title{Holi}

\date{\today}

% Contenido
\begin{document}
	\maketitle
	
	Sea \(x\) un número real positivo y sea \( y \in \mathbb{Z}\) tales que jljljjjllj
	\(
		x + y r 
		%		
		= 0,	
	\)
	para todo \( r \in \mathbb{R}\). Demostraremos que \(x\) es \(0\). Para ello notemos que si 
	\[
		x + yr = 0,
	\]
	para todo \( r \in \mathbb{R}\), entonces podemos tomar 
	\[
		r := -\dfrac{x}{y}	
	\]
	y [...]
	
	\vspace{\baselineskip}
	\begin{multicols}{2}
		Sea \(v\in M\) un vector de norma diferente de \(0\), buscamos todos los vectores tales que su producto punto con \(v\) sea \(1\); este \textbf{no es} el ortogonal de \(v\). Es más, dicho conjunto no es un espacio vectorial, es en efecto un espacio afín o un hiperplano. Este hiperplano es de la forma
		\[
			H = \{u \in M : \, v \cdot u = 1\}		.
		\]
	\end{multicols}
	
	
	Ahora vamos a resolver la ecuación diferencial:
	\[
		-\Delta u + u^2 = 0.	
	\]
	
	Pero una solución es \(0\), por lo tanto la EDP tiene infinitas soluciones. Es más, podríamos intentar con una función \(v\) y evaluar la siguiente expresión en la EDP:
	\[
			\dfrac{v^2}{v}.
	\]
	Esto no es lo mismo que \( \nicefrac{v}{2} \) o \( \dfrac{v}{2} \).
	
	Dejándo atrás las EDP's, recordemos que \( a_16^ja\) no es lo mismo que \( a_{16}^{ja}\), tampoco \( a_{16}\) y menos aún \( a^{ja}\).
	
	
	
	
	
	
	
	
	
	
	
	
	
	
	
	
	
	
	
	
	
	
	
\end{document}

 